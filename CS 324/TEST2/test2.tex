\documentclass[12pt]{article}

%adjustments
\parskip=.9ex
\textwidth=6in
\textheight=9in
\oddsidemargin=-0in
\topmargin=-0.75in

\begin{document}

\title{Summary of "An Improved Illumination Model for Shaded Display"}

\author{Tao Zhang\\
CS 324}

\maketitle

%=========================================

\section{Overview}
\paragraph{The illumination model is used to generate the color/light of an object’s surface at a given point on that surface. The article presents a new shading model that uses global information to calculate intensities that the traditional models can't deal with.}

\section{Key points}
\paragraph{Conventional Models uses Lamber's consine law as the simplest visible surface algorithms for shaders. Phong's Model is a more sophisticated model, but it has a drawback that has a big side effect on the quality of specular reflections. Another method that modeling an object's environment and mapping it onto a sphere of infinite radius was developed to partially solve the problem from Phong's model. }

\paragraph{Compared to the traditional model, the improved model added the diffuse component, which includes S, the intensity of light incident from the reflection direction, $k_t$, the transmission coefficient, and T, the intensity of light from the refraction direction. }


\paragraph{A tree of rays is used to make the illumination visible to the viewer.}

\section{New terminology/equation Encountered}
\paragraph{}

\section{Conclusion}
\paragraph{I get a big trouble now...}
\end{document}