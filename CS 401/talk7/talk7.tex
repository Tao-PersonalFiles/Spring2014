\documentclass[12pt]{article}

%adjustments
\parskip=.9ex
\textwidth=6.2in
\textheight=9in
\oddsidemargin=0in
\evensidemargin=0in
\topmargin=-.5in

\begin{document}
\paragraph{Tao Zhang\\Intellectual Property Law For Computer Science Students\\Dr. Annemarie Bridy\\April 7th, 2014}

\paragraph{Summary}

\begin{itemize}
\item In this lecture, Dr. Bridy introduced the issues in the law of intellectual property that computer programmers would be aware or interested at. 
\item Intellectual property(IP) rights are the legally recognized exclusive rights to creations of the mind. We have incentives for innovation and creativity, because ideas are nonrivalrous and nonexcludabe. As the law exists, it benefits the spurs creations, and the author get more payed.
\item Dr. Bridy introduced us four types of intellectual property: patents, copyright,  trademarks, and trade secrets. Each of them protect their specific things, and doesn't protect some other specific staff, also protect against something. The duration of IP rights are: 20 years for Patent, life of the author + 70 years for copyright, as long as the mark is source identifying used in commerce for trademarks, and as long as the secrets stays under wraps for trade secrets. 
\item As a computer science student, it is really important to know these laws as we don't want enrolled into any patent wars or some other issues on IP. 

\end{itemize}

\paragraph{This talk is really worth for students to learn something. I would like this talk return for the next seminar group.}
\end{document}