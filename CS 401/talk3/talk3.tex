\documentclass[12pt]{article}

%adjustments
\parskip=.9ex
\textwidth=6.2in
\textheight=9in
\oddsidemargin=0in
\evensidemargin=0in
\topmargin=-.5in

\begin{document}
\paragraph{Tao Zhang\\A Computational Scientist's Role in Current Biology\\Matt Settles\\Feb 24th, 2014}
\paragraph{Summary}
\begin{itemize}
	\item Bioinformatics has become an important part of biology. As massive amounts of experimental data is being generated, more researches, developments and applications of computational tools need to be done for bioinformatics.
	\item Bioinformatics is a biological data science that including Biologuy, Math, and Computer Science. The research area for iBest includes Microarraay Analysis, Structual Informatics(RNA,DNA), Hight-thoughtput, Computational Evelutionay, And Nucleatide Sequance Analysis. Matt showed us that the GenBank has grown from 1 million to 1 hundred billion since 1982. 
	\item Matt introduced one of the development tool Assembly by Reduced Complexity (ARC) at University of Idaho. It is a pipeline which facilitates iterative, reference guided de nono assemblies with the intent of:
	\begin{enumerate}
		\item Reducing time in analysis and increasing accuracy of results by only considering those reads which should assemble together
		\item Reducing reference bias as compared to mapping based approaches.
	\end{enumerate}
	The ARC is designed to work in situations where a whole-genome assembly is not the objective, but rather when the researcher wishes to assemble discreet 'targets' contaied withn next-generation shotgun sequence data. ARC decomplexifies the traditionally difficult problem of assembly by breaking the reads into small, manageable subsets which can then be assembled quickly and efficiently. Applications include those in which the researcher wishes to de novo assemble specific content and a set of semi-similar reference targets is available to initialize the assembly process.
	\item ARC has shown promise in: Assembly of bacterial plasmids; Assembly of viral genomes; Assembly of mitochondrial genomes using a distantly related reference; Assembly of exome capture data; Assembly of chloroplast genomes.
\end{itemize}

\paragraph{I did enjoy the talk. It is really interesting to know how biology and computer science can be combined and do some amazing work.}


\end{document}