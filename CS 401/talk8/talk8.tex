\documentclass[12pt]{article}

%adjustments
\parskip=.9ex
\textwidth=6.2in
\textheight=9in
\oddsidemargin=0in
\evensidemargin=0in
\topmargin=-.5in

\begin{document}
\paragraph{Tao Zhang\\Efficient Graph Matching Algortihms for QueryingBiological Networks and their Application to Web Service Composition\\Dr. Hasan Jamil\\April 14th, 2014}

\paragraph{Summary}

\begin{itemize}
\item Dr. Hasan Jamil presented that traditional models do not support graph representation, that RDBMS is flat and XML is a whole graph. They need to find some form of unit represantion that: Units can be processed one at a time in memory; Units can be stored and indexed; The global view of graphs can be recreated from the units.\\
\\
 Biological networks such as pathways and protein-protein interactions are naturally modeled using graphs. Searching for exact network is possible using identifiers or graph attribute match; however, a query graph is generally used to find matches that closely resemble the network. Such matching is fundamentally different from isomorphic matching of both attributed and non-attributed graphs. 
\item In the mathematical discipline of graph theory, a matching or independent edge set in a graph is a set of edges without common vertices. It may also be an entire graph consisting of edges without common vertices. Dr. Hasan Jamil introduced us a new approach to attributed graph matching using sub graph isomorphism with optimization opportunities.
\item Their research NetQL is still ongoing. It performs well on unattributed graphs that is XML based implementation, the In-memory algorithm is being implemented, and approximate matching is ongoing. AtoM is highly scalable and paraleeizable: Optimization is not currently possible but being investigated; MapReduce implementation underway.
\end{itemize}


\end{document}