\documentclass[12pt]{article}

%adjustments
\parskip=.9ex
\textwidth=6.2in
\textheight=9in
\oddsidemargin=0in
\evensidemargin=0in
\topmargin=-.5in

\begin{document}
\paragraph{Tao Zhang\\How Technology Molds our Assumptions\\Dr. Bob Rinker\\March 24th, 2014}

\paragraph{Summary}

\begin{itemize}
\item Dr. Rinker pointed that we use hardwares, but many of us don't know hardwares. So he showed some technology knowledge of the computer hardware from past.
\item Why Silicon? Dr. Rinker showed us the Periodic Table and talked about the conductors-Allow a generous flow of elements with very little applied forces-and the material with their relative conductivity. Semiconductors are poor. The effect of Dopant level on Resistivity: there are approximately $5 \times 10^{22}$ silicon atoms/$cm^3$. ``Pure" silicon usually contains 0.1-1 ppba of impurities. This corresponds to an impurity density of $0.5 - 5 \times 10^{13}$ atoms/cm.
\item Dr. Rinker also talked about the Junction Diode. Also, how photoresist Masking process work.
\begin{itemize}
	\item Grow $SiO_2$ Layer on Silicon
	\item (missed several steps)
	\item Etch $SiO_2$ with HF
	\item Remove photoresist
	\item Perform diffusion
	\item Remove $SiO_2$ Layer
\end{itemize}

\item What's more, Dr. Rinker talked about the TTL Gates(Transistor-transistor logic).

\end{itemize}
\end{document}