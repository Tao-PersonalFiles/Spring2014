\documentclass[12pt]{article}

%adjustments
\parskip=.9ex
\textwidth=6.2in
\textheight=9in
\oddsidemargin=0in
\evensidemargin=0in
\topmargin=-.5in

\begin{document}
\paragraph{Tao Zhang\\Agent-based modeling in philosophy\\Dr. Bert Baumgaertner\\March 24th, 2014}

\paragraph{Summary}

\begin{itemize}
\item Agent-based models are computational tools that can be used to study the emergence of population level patterns from interactions between individuals.
\item Dr. Baumgaertner shows us the population-level patterns. Some patterns are unique to collections of individuals: flocking, the wave, segregation, evolutionary processes, norms, distributed coordination/intelligence. There are several agent-based model: 
	\begin{itemize}
	\item individual or 'agents'
	\item crivtuall environment
	\item rules for how agents interact with each other and the environment.
	\end{itemize}		
\item Philosophers are interested in:
	\begin{itemize}
	\item Consciousness and how it emerges from the brain.
	\item Rationality and how it is shaped by evolution.
	\item Norms and the emergence of conventions, esp. Languages.
	\item Division of cognitive labour (the social aspects of science)
	\item Spread of opinions, belief, and knowledge.
	\end{itemize}
\item Dr. Baumgaertner says that they should preserve (even encourage) dissenting opnions. Not just for the sake of individuals, but for the sake of the whole. Diversity of opinions helps them get to the truth, and get to the future.
\end{itemize}

\end{document}