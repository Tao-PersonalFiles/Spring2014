\documentclass[12pt]{article}

%adjustments
\parskip=.9ex
\textwidth=6.2in
\textheight=9in
\oddsidemargin=0in
\evensidemargin=0in
\topmargin=-.5in

\begin{document}
\paragraph{Tao Zhang\\Combining DNA fingerprints to dissect microbial communities\\Dr. James Foster\\May 5th, 2014}

\paragraph{Summary}

\begin{itemize}
\item It is hard to dissect microbial communities. 

\item Human is one out of 2 million named species. There are ten times more bacterial than human cells in your body. Up to 1 Billion species, there's only about 5,000 are known, and approximate only 3\% of bacteria can be grown in the lab.

\item Dr. James pointed out that there is one technique can distinguish one species from another, called "fingerprint". They developed a technique for combining DNA sequnce data from multiple fingerprints, while the current techniques can only use one single fingerprint. He talked about high throughput fingerprinting:
	\begin{itemize}
	\item Get "every" DNA molecule in a sample: break cells up, wash, filter
	\item Isolate fingerprint regions from all bacteria
	\item Sequence them all
	\end{itemize}
There will be 10-20million fingerprint sequences. Then infer how many of which species were there. 

\item To interpret fingerprint data:
\begin{itemize}
\item Compute similarity between fingerprint sequences
\item Cluster, call a cluster a ``species"
\item Number of clusters is species richness
\item Size of clusters species abundance
\end{itemize}


\end{itemize}
\end{document}