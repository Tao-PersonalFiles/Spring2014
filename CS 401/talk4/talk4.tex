\documentclass[12pt]{article}

%adjustments
\parskip=.9ex
\textwidth=6.2in
\textheight=9in
\oddsidemargin=0in
\evensidemargin=0in
\topmargin=-.5in

\begin{document}
\paragraph{Tao Zhang\\COTSBOTs: Commodity-Off-The-Shelf(COTS)Robots\\Dr. Terry Soule\\March 3rd, 2014}

\paragraph{Summary}

\begin{itemize}
	\item For this topic, Dr. Soule introduced us about how to leverage the advantages of cost, reliabilitym availability, and features in design of Cotsbots(robots) with mass produced devices/parts.
	\item Cots bots are robots built according to the Commercial-Off-The-Shelf desin principle, which is that design using readily avaibable, commercial products, rather than specialized custom designed parts. Mostly, the design's cost is less than \$500. 
	\item The basic design is that:
	\begin{enumerate}
		\item The 'brains' - typically a smart phone or netbook, which makes decision, learn, etc.
		\item The 'body' - the platform from a RC vehicle.
		\item The 'spinal cord' - a microcontroller or motor controller, such as made by Arduino or Phidgets.
	\end{enumerate}
	The brains control the robot, the body moves the robot, and the spinal cord transmits messages between the two through USB or Bluetooth. As those two robots Dr. Soule showed to us, one robot is controlled by the app through another android phone; another robot will trace the green ball via the camera of the smart phone brain, (once the 'brain' lost the target, it will turn around to find the target). 
	\item The robot also has the learning system by giving it some operations. Then the 'brian' will store and send the operations to another robot via a server? Then the robot will keep doing the operations, for example driving on blue 'road'.
	\item In short, the robot looks really cool and it really doesn't cost too much. 
\end{itemize}

\paragraph{I really enjoy the talk and want this talk to return for the next seminar group. If possible, I will go and attend the group, or even made one by my own. Pretty cool!}
\end{document}