\documentclass[12pt]{article}

\usepackage{amscd}
\usepackage{amsfonts}
\usepackage{amsmath}
\usepackage{amssymb}
\usepackage{graphicx}
\usepackage{c}

%adjustments
\parskip=.9ex
\textwidth=7in
\textheight=9in
\oddsidemargin=-.25in
\topmargin=-.75in

\begin{document}

\title{Project 1B}

\author{Tao Zhang\\
CS 472}

\maketitle
\newpage


%=====================================================
%			Descrition of Genetic Algorithm (GA)
%=====================================================
\section{Descrition of Genetic Algorithm (GA)}
\paragraph{Generally, my Genetic Algorithm used Steady-State type that: } 
\begin{itemize}
	\item generate a population
	\item have enough loops
	\item for each loop 
	\begin{enumerate}
		\item tournament select two parents
		\item copy the gene to 2 children
		\item mutate 2 children
		\item replace two loser (tournament selection) by two children
	\end{enumerate}
	\item finally check the fitness of each individual and find the best
\end{itemize}

\paragraph{My code for the GA so far has these functions:}
\begin{itemize}
	\item int main(): just a main function, which will present the final result.
	\begin{lstlisting}
/*
 * #include <...>
 */

// #define bunch of constant values

int main()
{
  GeneticAlgorithm();
 
  // print solution
  // print Best fitness 
 
  return 0;
}
\end{lstlisting}

	\item double GeneticAlgorithm(): The major algorithm part. It will finally return the best fitness, also pass the reference of the best individual (solution) so far.
	\begin{lstlisting}
	/* Steady-State */
double GeneticAlgorithm(double resultA[X_I])
{
  // Generate a population
  for 0 to Population step 1
    // Generate each individual 
    for 0 to 30 step 1
      // Take random values in range for each vector
    // Calculate each individuals fitness, put into an array

  while(count < population*100) // as Dr. Soul suggested
  {
    // select father by Tournament
    // select mother by Tournament
  	
	/* crossover is not need right now, 
	 * but I just make it as copy the same gene
	 */
    // copy father to born child1
    // copy mother to born child2  
    
    // Mutate both children	
    
    // select two losers by Tournament
    // Replace two losers by two children
  }
  
  /* Finally, check the fitness array */
  bestEval <- 9999(inital);
  for 0 to population step 1
    if fitness[i] < bestEval
      bestEval <- fitness[i];
      best = i	// record the position;
      
  // copy best solution to resultA      
      
  return bestEval;
}
	\end{lstlisting}
	
	\item double evaluate(): just the function to calculate the fitness.
	
	\item int selection(): so this a tournament selection function which select a winner/loser (depend on the index) from a sample of N random individuals (N is also a random number). Finally the function returns a int which is the index of that individual among population array.
	\begin{lstlisting}
	/* Tournament selection */
int selection(int good_poor, double fitness[POP]) 
/* good_poor is the index to determe we are select a winner or loser*/
{
  // get a random winner and calc his fitness
  winner <- rand()%POP;
  winner_fitness <- fitness[winner];
  
  // generate a random N
  N <- rand()%POP ;
  // loop to get the best
  for 0 to N step 1
    temp <- rand%POP
    if(good_poor == GOOD && fitness[temp] < winner_fitness){		
    // this is for winner
      winner_fitness <- fitness[temp];
      winner <- temp;
    }else if(good_poor == POOR && fitness[temp] > winner_fitness){	
    // this is for loser
      winner_fitness <- fitness[temp];
      winner <- temp;
    }
    
    return winner; // the index among population
}
	\end{lstlisting}
	
	\item void mutate(): Mutate the children by ``creep". For each vecter in child, we have 50percent chance to mutate the value by +/- a small value (I did 0.01).
	\begin{lstlisting}
void mutate(double child[X_I])
{
  for 0 to 30 step by 1
    if rand()%10 < 5  // 50% chance
      // 50% + and 50% -
      child[i] += (rand()%10 < 5 ? -1 : 1)*INTERVAL;
}
	\end{lstlisting}
	
\end{itemize}

%=====================================================
%			Results
%=====================================================
\section{Results}
	\begin{center}
	\begin{tabular}{| l | r | r | r | r | }
	\hline
	\bf GA Sphere & \bf 1 & \bf 2 & \bf 3 & \bf 4 \\
	\hline
	Population & 100 & 1000 & 1000 & 1000 \\
	Mutation Interval & 0.01 & 0.01 & 0.01 & 0.001 \\
	Running Times(*POP) & 100 & 100 & 500 & 100 \\
	First Best Fitness & 473.867100 & 509.951500 & 548.528900 & 537.843500 \\
	Final Best Fitness & 0.001300 & 0.000600 & 0.000500 & 0.000009 \\
	real time & 0m0.062s & 0m3.544s & 0m17.745s & 0m3.548s \\
	\hline
	\end{tabular}
	\end{center}
 
\paragraph{I tested 4 situations:}
\begin{itemize}
	\item 100 population with running times as 100 times the amount of population
	\item 1000 population with running times as 100 times the amount of population
	\item 1000 population with running times as 500 times the amount of population
	\item and 1000 population with running times as 100 times the amount of population, but with Mutation interval of 0.001
\end{itemize}
\paragraph{My codes work very well I think.}

\section{Concluion}
\paragraph{The results are good enough to go. As the population and running times go higher, the more accurate answer we will get. Also, I can narrow my mutation inverval, which can make the final fitness much more accurate.}

\end{document}