\documentclass[12pt]{article}

\usepackage{amscd}
\usepackage{amsfonts}
\usepackage{amsmath}
\usepackage{amssymb}
\usepackage{graphicx}
\usepackage{c}

%adjustments
\parskip=.9ex
\textwidth=7in
\textheight=9in
\oddsidemargin=-.25in
\topmargin=-.75in

\begin{document}

\title{Project 1 - Genetic Algorithm}

\author{Tao Zhang\\
CS 472}

\maketitle
\newpage

%====================================

\begin{abstract}
My Genetic Algorithm works well for most functinos, except Rastrigin and Schwefel function. I tested all the crossover methods, except ``Permutation". Among the crossover methods, I find that 1 point crossover works the best for most of the functions. 
\end{abstract}

\section{Description for Genetic Algorithm}
\paragraph{Generally, my Genetic Algorithm used Steady-State type that: } 
\begin{itemize}
	\item generate a population
	\item have enough loops
	\item for each loop 
	\begin{enumerate}
		\item tournament select two parents
		\item crossover (mainly by 1 point crossover)
		\item mutate 2 children
		\item replace two loser (tournament selection) by two children
	\end{enumerate}
	\item finally check the fitness of each individual and find the best
\end{itemize}

\subsection{Fitness function: just math, no need to explain}
\begin{itemize}
	\item Sphere:
	\begin{lstlisting}
double evaluate(double x[X_I])	
{
	double z = 0;
	double tmp;
	int i;
	for(i = 0 ; i < X_I; i++ )
	{
		tmp = x[i];
		z += tmp*tmp;
	}
    
	return z;
}
	\end{lstlisting}
	
	\item Rosenbrock
	\begin{lstlisting}
double evaluate(double x[X_I])
{
	double z = 0;
	int i;
	for(i = 0; i < X_I - 1; i++) {
		z += 100*pow(x[i+1]-pow(x[i],2) , 2) + pow(x[i] - 1, 2);
	}

	return z;
}
	\end{lstlisting}
	
	\item Rastrigin
	\begin{lstlisting}
double evaluate(double x[X_I])
{
	double z = 10*X_I;
	int i;
	for(i = 0; i < X_I; i++){
		z += pow(x[i] , 2) - 10*cos(2*M_PI*x[i]);
	}
	return z;
}
	\end{lstlisting}
	
	\item Schwefel
	\begin{lstlisting}
double evaluate(double x[X_I])
{
	double z = 418.9829*X_I;
	int i;
	for(i = 0; i < X_I; i++){
		z += x[i]*sin(sqrt(abs(x[i])));
	}
	return z;
}
	\end{lstlisting}
	
	\item Ackley
	\begin{lstlisting}
double evaluate(double x[X_I])
{
	double z = 20+exp(1);
	double a = 0;
	double b = 0;
	int i;
	for(i = 0; i<X_I;i++){
		a += pow(x[i] , 2);
		b += cos(2*M_PI*x[i]);
	}
	z = z - 20*exp(-0.2*sqrt(a/X_I)) - exp(b/X_I);

	return z;
}
	\end{lstlisting}
	
	\item Griewangk
	\begin{lstlisting}
double evaluate(double x[X_I])
{
	double z = 1;
	double a = 0;
	double b = 1;
	int i;
	for(i = 0; i < X_I; i++) {
		a += pow(x[i],2) / 4000.00;
		b *= cos(x[i]/(i+1));
	}
	z = z + a - b;

   	return z;
}
	\end{lstlisting}
\end{itemize}

\subsection{Generate initial random solutions}
\begin{itemize}
	\item 
	\begin{lstlisting}
  // Generate a population
  for 0 to Population step 1
    // Generate each individual 
    for 0 to 30 step 1
      // Take random values in range for each vector
    // Calculate each individuals fitness, put into an array
	\end{lstlisting}
\end{itemize}

\subsection{Crossover}
\paragraph{Once we get the parents, we will genearte two children by crossover. I did 4 crossover methods, listed below:}
\begin{itemize}
	\item one point crossover
	\begin{lstlisting}
void crossover(double father[X_I], 
		double mother[X_I], double children[2][X_I])
{
	int cut = rand()%X_I;	// get a random cut/point 
	int i;
	for(i = 0; i < cut; i++){	// keep the front half
		children[0][i] = father[i];
		children[1][i] = mother[i];
	}
	for( ; i < X_I; i++){		// swap/cross the back half
		children[1][i] = father[i];
		children[0][i] = mother[i];
	}
}
	\end{lstlisting}
	
	\item two points crossover
	\begin{lstlisting}
void crossover(double father[X_I], 
		double mother[X_I], double children[2][X_I])
{
	// get two random cuts/points
	int cut1 = rand()%X_I;
	int cut2 = rand()%(X_I - cut1);
	int i;
	for(i = 0; i < cut1; i++){	
		children[0][i] = father[i];
		children[1][i] = mother[i];
	}
	for( ; i < cut1+cut2; i++){	// only swap the middle
		children[1][i] = father[i];
		children[0][i] = mother[i];
	}
	for( ; i < X_I; i++){
		children[0][i] = father[i];
		children[1][i] = mother[i];
	}
}
	\end{lstlisting}
	
	\item uniform crossover
	\begin{lstlisting}
void crossover(double father[X_I], 
		double mother[X_I], double children[2][X_I])
{
	double tmp= 0;
	int P = 0;
	int i;
	for(i = 0; i < X_I; i++) {
		// for each value, we have 30% chance to swap that "gene"
		if((P = rand()%100) < 30) {
			tmp = children[0][i];
			children[0][i] = children[1][i];
			children[1][i] = tmp;
		}
	}
}
	\end{lstlisting}
	
	\item arithmetic
	\begin{lstlisting}
void crossover(double father[X_I], 
		double mother[X_I], double children[2][X_I])
{
	double tmp = 0;
	int i;
	for(i = 0; i < X_I; i++) {
		// get the avg of two values and get two same children
		tmp = (children[0][i] + children[1][i]) / 2;
		children[0][i] = tmp;
		children[1][i] = tmp;
	}
}
	\end{lstlisting}
\end{itemize}

\subsection{Mutation}
\paragraph{Mutate the children by ``creep". For each vecter in child, we have 50percent chance to mutate the value by +/- a small value (0.001 for Rosenbrock; 0.01 for Sphere, Rastrigin; 0.1 for Ackley; 1 for Schwefel, Griewangk).}
\begin{itemize}
	\item 
	\begin{lstlisting}
void mutate(double child[X_I])
{
  for 0 to 30 step by 1
    if rand()%10 < 5  // 50% chance
      // 50% + and 50% -
      child[i] += (rand()%10 < 5 ? -1 : 1)*INTERVAL;
}
	\end{lstlisting}
\end{itemize}

\subsection{GeneticAlgorithem}
\paragraph{The major algorithm part. It will finally return the best fitness, also pass the reference of the best individual (solution) so far.}
\begin{itemize}
	\item
	\begin{lstlisting}
	/* Steady-State */
double GeneticAlgorithm(double resultA[X_I])
{
  // Generate a population
  // Calculate each individuals fitness, put into an array

  while(count < population*100) // as Dr. Soul suggested
  {
    // select father by Tournament
    // select mother by Tournament (diff from father)
  	
	// crossover father and mother to get two children
    
    // Mutate both children	
    
    // select two losers by Tournament
    // Replace two losers by two children
  }
  
  /* Finally, check the fitness array */
  bestEval <- fitness[0](inital);
  for 0 to population step 1
    if fitness[i] < bestEval
      bestEval <- fitness[i];
      best = i	// record the position;
      
  // copy best solution to resultA      
      
  return bestEval;
}
	\end{lstlisting}
\end{itemize}

\subsection{Selection}
\paragraph{This is a tournament selection function which select a winner/loser (depend on the index) from a sample of N random individuals (N is also a random number). Finally the function returns a int which is the index of that individual among population array.}
\begin{itemize}
\item 
	\begin{lstlisting}
	/* Tournament selection */
int selection(int good_poor, double fitness[POP]) 
/* good_poor is the index to determe we are select a winner or loser*/
{
  // get a random winner and calc his fitness
  winner <- rand()%POP;
  winner_fitness <- fitness[winner];
  
  // generate a random N
  N <- rand()%POP ;
  // loop to get the best
  for 0 to N step 1
    temp <- rand%POP
    if(good_poor == GOOD && fitness[temp] < winner_fitness){		
    // this is for winner
      winner_fitness <- fitness[temp];
      winner <- temp;
    }else if(good_poor == POOR && fitness[temp] > winner_fitness){	
    // this is for loser
      winner_fitness <- fitness[temp];
      winner <- temp;
    }
    
    return winner; // the index among population
}
	\end{lstlisting}
\end{itemize}


%=============================
%		Results
%=============================
\section{Results}
\subsection{Sphere}
\begin{center}
	\begin{tabular}{| l | r | r | r | r | }
	\hline
	\bf Crossover & \bf 1point & \bf 2points & \bf uniform & \bf  arithmetic \\
	\hline
	Population & 50 & 50 & 50 & 50 \\
	Mutation Interval & 0.01 & 0.01 & 0.01 & 0.01 \\
	Running Times(*POP) & 100 & 100 & 100 & 100 \\
	1st Best Fitness & 187.526800 & 148.298600 & 142.681100 & 148.242200 \\
	1st Avg Fitness & 258.182810 & 259.975670 & 258.766226 & 265.070316 \\
	Final Best Fitness &  0.002500 & 0.001200 & 2.557200 & 1.217325 \\
	Final Avg Fitness & 0.003112 & 0.001756 & 5.150842 &  2.173404 \\
	\hline
	\end{tabular}
	\end{center}
	
\subsection{Rosenbrock}
\begin{center}
	\begin{tabular}{| l | r | r | r | r | }
	\hline
	\bf Crossover & \bf 1point & \bf 2points & \bf uniform & \bf  arithmetic \\
	\hline
	Population & 100 & 100 & 100 & 100 \\
	Mutation Interval & 0.001 & 0.001 & 0.001 & 0.001\\
	Running Times(*POP) & 2000 & 2000 & 2000 & 2000 \\
	1st Best Fitness & 6633.415331 & 6537.239168 & 6105.194947 & 5716.116572 \\
	1st Avg Fitness & 14865.619920 & 14462.804456 & 14392.547199 & 14755.587222 \\
	Final Best Fitness & 0.192033 & 4.180114 &  159.677838 & 141.739331 \\
	Final Avg Fitness & 0.196815 & 4.183143 & 209.094239 & 148.872279 \\
	\hline
	\end{tabular}
	\end{center}
	
\subsection{Rastrigin}
\paragraph{My genetic algorithm doesn't work for Rastrigin function. The best result I get will stay (little change) at some point, even I add more running times. But the fitness function I write should be correct. I have no idea about it.}
	
\subsection{Schwefel}
\paragraph{For this function, I get the same trouble as Rastrigin function.}
	
\subsection{Ackley}
\begin{center}
	\begin{tabular}{| l | r | r | r | r | }
	\hline
	\bf Crossover & \bf 1point & \bf 2points & \bf uniform & \bf  arithmetic \\
	\hline
	Population & 50 & 50 & 50 & 50\\
	Mutation Interval & 1 & 1 & 1 & 1\\
	Running Times(*POP) & 1000 & 1000 & 1000 & 1000 \\
	1st Best Fitness & 18.924080 & 18.885888 & 18.775362 & 18.762705 \\
	1st Avg Fitness & 19.334254 & 19.379073 & 19.375065 & 19.308400\\
	Final Best Fitness & 1.711187 & 1.841785 & 2.637531 & 2.467152\\
	Final Avg Fitness & 2.319135 & 2.409010 & 14.853659 & 15.595730 \\
	\hline
	\end{tabular}
	\end{center}
	
\subsection{Griewangk}
\begin{center}
	\begin{tabular}{| l | r | r | r | r | }
	\hline
	\bf Crossover & \bf 1point & \bf 2points & \bf uniform & \bf  arithmetic \\
	\hline
	Population & 50 & 50 & 50 & 50\\
	Mutation Interval & 1 & 1 & 1 & 1 \\
	Running Times(*POP) & 100 & 100 & 100 & 100 \\
	1st Best Fitness & 533.451750 & 612.249000 & 594.165750 & 518.436000 \\
	1st Avg Fitness & 929.957840 & 915.618320 & 906.391795 & 900.314490 \\
	Final Best Fitness & 0.277369 & 2.207000  & 6.057750 & 4.453250 \\
	Final Avg Fitness & 0.391046 & 2.354080 & 11.263200 & 10.335357\\
	\hline
	\end{tabular}
	\end{center}
	
\section{Conclusion}
My Genetic Algorithm should be successful, but still need more researches to figure out the trouble I get on Rastrigin and Schwefel functions.
\end{document}