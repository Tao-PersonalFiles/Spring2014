\documentclass[12pt]{article}

\usepackage{amscd}
\usepackage{amsfonts}
\usepackage{amsmath}
\usepackage{amssymb}
\usepackage{graphicx}
\usepackage{c}

%adjustments
\parskip=.9ex
\textwidth=7in
\textheight=9in
\oddsidemargin=-.25in
\topmargin=-.75in

\begin{document}

\title{SubProject 2B - Genetic Programming}

\author{Tao Zhang\\
CS 472}

\maketitle
\newpage

%====================================

\begin{abstract}
For this subproject, I write the conditional function as steady state, a Mutation function, Crossover function with two trees, and a Tournament selection function. Every function works well.
\end{abstract}

\section{Description}
	\subsection{Conditional}
	\paragraph{My conditional function is steady state, and here's the brief description.}
	\begin{itemize}
	\item
	\begin{lstlisting}
    public void steady_state(){
        // initialize all the index for parents and losers
        // initialize son and dau as individual
        
        int count = 0;
        while(count < pop_size*100){
            father = tournament(good);
            while((mother = tournament(good)) == father){
                // do nothing
                // just make sure father != mother
            }
            
            // crossover to get children
            crossover(son, dau);
                    
            // mutate
            son.Mutation();
            dau.Mutation();
            
            // find two losers
            loser1 = tournament(bad);
            while((loser2 = tournament(bad)) == loser1){
                // do nothing
                // just make sure loser1 != loser2
            }
            // replace two losers by chilren
            
            // refresh the data: size, fitness
        }
    }
	\end{lstlisting}
	\end{itemize}
	
	\subsection{Selection}
	\paragraph{I choose tournament selection method, which the function will return the winner index.}
	\begin{itemize}
	\item
	\begin{lstlisting}
    public int tournament(boolean good){
        int winner = randomGenerator.nextInt(pop_size);
        double winner_fitness = fitness[winner];
        int tmp;
        
        int i;
        int N = pop_size/3;
        for(i = 0; i < N; i++){
            tmp = randomGenerator.nextInt(pop_size);
            if(good == true && fitness[tmp] > winner_fitness){
                winner_fitness = fitness[tmp];
                winner = tmp;
            }else if(good == false && fitness[tmp] < winner_fitness){
                winner_fitness = fitness[tmp];
                winner = tmp;
            }
        }
        
        return winner;
    }
	\end{lstlisting}
	\end{itemize}
	
	\subsection{Mutation}
	\paragraph{My mutation function will generate a random number (0~size), and track the tree while counting until the counter = random number. Delete that subtree's left and right, then use full method to regenerate the subtree.}
	\begin{itemize}
	\item
	\begin{lstlisting}
    public void Mutation(){
        int r = randomGenerator.nextInt(terms+non_terms);
        
        Node sub;
        resetCount();
        // the TrackRoot function will return the node we point to
        sub = TrackRoot(root, r);
        //delete the left and right sub trees
        
        int max = randomGenerator.nextInt(5);
        // regenerate the subtree with another random max depth full
        full(0, max, sub);
    }
	\end{lstlisting}
	\end{itemize}
	
	\subsection{Crossover}
	\paragraph{My crossover function will read individuals: son and dau, which are the copy of parents. Inside the function:}
	\begin{itemize}
	\item
	\begin{lstlisting}
    public void crossover(Individual son, Individual dau){
        Node gene_of_son = new Node();
        Node gene_of_dau = new Node();
        
        // set random gene position for parents
        int r_son, r_dau;
        
        boolean rule = false;
        
        while(rule != true){
            // generate random position of father's gene
            r_son = randomGenerator.nextInt(son.terms + son.non_terms);
            // point to random gene
            son.resetCount();
            gene_of_son = son.TrackRoot(son.root, r_son);
            if apply 90/10 rule
            	rule = true
        }
        
        rule = false;
        // same way to generate gene_of_dau
        
        // put father's gene into another tmp individual
        Individual tmp_gene = new Individual();
        tmp_gene.copyNode(tmp_gene.root, gene_of_son);
        
        // delete father's subgene
        
        // copy mother's gene onto father
        son.copyNode(gene_of_son, gene_of_dau);
        
        // delete mother's subgene
        
        // copy tmp individual (father's gene) onto mother
        dau.copyNode(gene_of_dau, tmp_gene.root);
    }
	\end{lstlisting}
	\end{itemize}
	
\section{Results}
\paragraph{I tested all the functions seperately. \\The mutation function does generate a new subtree.}
\begin{itemize}
\item Before mutation: X + X + X / X - 0.00 * 6.00 - X / X = ?
\item After Mutation: X + X + X / X - 0.00 * X * X - X + X - X / X = ?
\end{itemize}

\paragraph{The crossover function does swap the subtree of son and dau}
\begin{itemize}
\item Before Crossover:\\father: 9.00 + 1.00 - 1.00 * X / 8.00 * X / X * X - X - X - X / X * 3.00 + X + 4.00 - 0.00 = ?\\
mother: 2.00 - X * X / 1.00 + 7.00 * 3.00 / X * X = ?
\item After Crossover: \\son: 9.00 + 1.00 - 1.00 * X / 8.00 * X / X * X - X - X - X / X * X = ?\\
dau: 2.00 - X * X / 1.00 + 7.00 * 3.00 / 3.00 + X + 4.00 - 0.00 * X = ?
\end{itemize}

\paragraph{The Conditional function and tournament function also works well. }

\section{Conclusion}
\paragraph{Every function works well, but I will come to have some questions after break.}
\end{document}